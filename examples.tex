\documentclass{documentation}
\usepackage{syntax}

\begin{document}
\begin{titlepage}
\begin{center}
Studencka Pracownia Inżynierii Oprogramowania\\
Instytutu Informatyki Uniwersytetu Wrocławskiego\\[6cm]

Damian Bogel, Łukasz Dąbek\\[1cm]
\textsc{\LARGE Kompilator Języka Gordon}\\[0.5cm]
\textsc{\large Przykładowe programy}

\vfill
Wrocław 2014 \\[2.5cm]

\end{center}
\end{titlepage}

\newpage
\begin{table}
	\centering
    \captionsetup{name=Tabela,labelsep=period}
	\caption{Historia zmian w dokumencie}
		\begin{tabular}{|r|c|c|l|l|}
		\hline
		Lp.  & Data       & Nr wersji & Autor                 & Zmiana \\ \hline
		1.   & 2014-01-10 & 0.01 & Damian Bogel & Utworzenie dokumentu \\ \hline
	\end{tabular}
\end{table}
\newpage

\tableofcontents
\setcounter{page}{2}

\newpage

\section{Przykłady}
\noindent W kolejnych podrozdziałach przedstawione zostały przykładowe programy
napisane w~ języku \textsc{Gordon}.

\subsection{Liczby Fibonacciego}
\noindent Poniższy program przedstawia funkcję, która oblicza czterdziestą
liczbę Fibonacciego.

\begin{verbatim}
let fib n = 
    if n = 1
    then 1
    else fib (n - 1) + fib (n - 2)

fib 40
\end{verbatim}

\subsection{Uruchamianie wątków}
\noindent W nowym wątku można uruchomić dowolne wyrażenie poprzedzając je słowem kluczowym
\texttt{go}. Po wykonaniu poniższego programu, zmienna \texttt{x} może mieć wartość 2 lub 3.

\begin{verbatim}
let x = 0
go (x := 2)
go (x := 3)
\end{verbatim}


\subsection{Problem producentów i konsumentów}
\noindent Rozwiązanie klasycznego problemu za pomocą mechanizmów synchronizacji
oferowanych przez język \textsc{Gordon}.

\begin{verbatim}
letatomic buffer = Array Int 10
letatomic buf_count = 0
letatomic index = 0

let read buffer buf_count index = 
    let val = Int

    letatomic try_read =
        if buf_count > 0
        then
            buf_count = buf_count - 1
            val := buffer[index]
            index = (index + 1) mod 10
            return true
        else
            return false
        
    while call try_read == false do
        continue

let write buffer buf_count index val =
    letatomic try_write =
        if buf_count < 10
        then
            buffer[(index + buf_count) mod 10] := val
            buf_count := buf_count + 1
            return true
        else
            return false

    while call try_write == false do
        continue

\end{verbatim}


\begin{verbatim}

\end{verbatim}


\end{document}
