\documentclass{documentation}

\begin{document}
\begin{titlepage}
\begin{center}
Studencka Pracownia Inżynierii Oprogramowania\\
Instytutu Informatyki Uniwersytetu Wrocìawskiego\\[6cm]

Damian Bogel, Łukasz Dąbek\\[1cm]
\textsc{\LARGE Kompilator Języka Gordon}\\[0.5cm]
\textsc{\large Oszacowanie kosztów projektu}

\vfill
Wrocław 2013 \\[2.5cm]

\end{center}
\end{titlepage}

\newpage
\begin{table}
	\centering
	\caption{Historia zmian w dokumencie}
		\begin{tabular}{|r|c|c|l|l|}
		\hline
		Lp.  & Data       & Nr wersji & Autor                 & Zmiana \\ \hline
		1.   & 2013-11-05 & 0.01 & Łukasz Dąbek & Utworzenie dokumentu \\ \hline
	\end{tabular}
\end{table}
\newpage

\tableofcontents
\setcounter{page}{2}

\newpage

\section{Wstęp}
\noindent W celu oszacowania kosztów grono niezależnych ekspertów zostało poproszonych
o oszacowanie kosztów projektu. Rzeczone szacunki prezentujemy poniżej.

\section{Szacunki ekspertów}
\subsection{Simon Marlow}
\noindent Simon Marlow to jeden z twórców kompilatora \textsc{GHC}, znany ekspert
z dziedziny języków programowania. Aktualnie w Facebooku zajmuje się programowaniem
współbieżnym i równoległym. Poniżej przytaczamy fragment rozmowy z panem Marlowem:
\begin{quotation}
    Nie jest to trywialny projekt. Koszta szacuję na ok. 200000 zł.
\end{quotation}

\subsection{Simon Peyton Jones}
Simon Payton Jones jest informatykiem pracującym w Wielkiej Brytanii. Zajmuje się zastosowaniami
oraz implementacją języków funkcyjnych, jest jednym z twórców kompilatora \textsc{GHC}.
Pan Jones oszacował koszty projaktu na 170000zł.

\subsection{Haskell Curry}
Haskell Curry jest światowym ekspertem w dziedzinie logiki i języków programowania.
Zainspirował on pracę nad językami programowania \textsc{Haskell} i \textsc{Curry}.
Obecnie nie żyje. Oszacował koszty projektu na 180000zł.

\section{Podsumowanie}
Na podstawie danych z kosmosu i zmyślnonych opinii ekspertów szacujemy koszta na
$2^{20}$ zł.

\end{document}
