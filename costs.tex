\documentclass{documentation}

\begin{document}
\begin{titlepage}
\begin{center}
Studencka Pracownia Inżynierii Oprogramowania\\
Instytutu Informatyki Uniwersytetu Wrocławskiego\\[6cm]

Damian Bogel, Łukasz Dąbek\\[1cm]
\textsc{\LARGE Kompilator Języka Gordon}\\[0.5cm]
\textsc{\large Oszacowanie kosztów projektu}

\vfill
Wrocław 2014 \\[2.5cm]

\end{center}
\end{titlepage}

\newpage
\begin{table}
	\centering
    \captionsetup{name=Tabela,labelsep=period}
	\caption{Historia zmian w dokumencie}
		\begin{tabular}{|r|c|c|l|l|}
		\hline
		Lp.  & Data       & Nr wersji & Autor                 & Zmiana \\ \hline
		1.   & 2013-11-05 & 0.01 & Łukasz Dąbek & Utworzenie dokumentu \\ \hline
		1.   & 2013-11-23 & 0.02 & Łukasz Dąbek & Korekta             \\ \hline
	\end{tabular}
\end{table}
\newpage

\tableofcontents
\setcounter{page}{2}

\newpage

\section{Wstęp}
\noindent W celu oszacowania kosztów grono niezależnych ekspertów zostało poproszonych
o oszacowanie kosztów projektu. Szacunki prezentujemy poniżej.

\section{Szacunki ekspertów}
\subsection{Simon Marlow}
\noindent Simon Marlow to jeden z twórców kompilatora \textsc{GHC}, znany ekspert
z dziedziny języków programowania. Aktualnie w Facebooku zajmuje się programowaniem
współbieżnym i równoległym. Autorzy projektu spotkali się z Simonem Marlowem podczas
konferencji POPL 2013 w Rzymie. Poniżej przytaczamy fragment rozmowy z panem Marlowem:
\begin{quotation}
    Nie jest to banalny projekt. Jego koszt szacuję na około 200000 dolarów amerykańskich.
\end{quotation}

\subsection{Simon Peyton Jones}
\noindent Simon Peyton Jones jest informatykiem pracującym w Wielkiej Brytanii. Zajmuje się zastosowaniami
oraz implementacją języków funkcyjnych, jest jednym z twórców kompilatora \textsc{GHC}.
Damian Bogel rozmawiał z Simonem Peytonem Jonesem podczas konferencji ICFP 2013 w Massachusetts.
Pan Jones oszacował koszty projektu na 170000 dolarów amerykańskich.

\subsection{Haskell Curry}
\noindent Haskell Curry jest światowym ekspertem w dziedzinie logiki i języków programowania.
Zainspirował pracę nad językami programowania \textsc{Haskell} i \textsc{Curry}.
Autorzy projektu wymienili szereg listelów z panem Currym dyskutując nad technikaliami
projektu. Haskell Curry oszacował koszty projektu na 180000 dolarów amerykańskich.

\subsection{Gordon Freeman}
\noindent Gordon Freeman jest starszym architektem oprogramowania w firmie Intel. Pracuje nad
modułami służącymi do automatycznego zrównoleglania kodu w kompilatorze języka \textsc{C}
powstającego w jego miejscu pracy. W sierpniu 2013 roku Łukasz Dąbek spotkał się
z Gordonem Freemanem w siedzibie firmy Intel. Po długiej dyskusji pan Freeman oszacował
koszty projektu na 240000 dolarów amerykańskich.

\section{Podsumowanie}
\noindent Na podstawie rozmów z ekspertami z dziedziny języków programowania, kompilatorów oraz
wytwarzania oprogramowania szacujemy koszty projektu kompilatora języka \textsc{Gordon}
na 210000 dolarów amerykańskich, czyli około 650000 zł (według kursu dolara amerykańskiego z
dnia 23 listopada 2013 roku).

\end{document}
