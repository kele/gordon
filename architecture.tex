\documentclass{documentation}

\begin{document}
\begin{titlepage}
\begin{center}
Studencka Pracownia Inżynierii Oprogramowania\\
Instytutu Informatyki Uniwersytetu Wrocławskiego\\[6cm]

Damian Bogel, Łukasz Dąbek\\[1cm]
\textsc{\LARGE Kompilator Języka Gordon}\\[0.5cm]
\textsc{\large Architektura kompilatora}

\vfill
Wrocław 2013 \\[2.5cm]

\end{center}
\end{titlepage}

\newpage
\begin{table}
	\centering
    \captionsetup{name=Tabela}
	\caption{Historia zmian w dokumencie}
		\begin{tabular}{|r|c|c|l|l|}
		\hline
		Lp.  & Data       & Nr wersji & Autor                 & Zmiana \\ \hline
		1.   & 2013-10-22 & 0.01 & Łukasz Dąbek & Utworzenie dokumentu \\ \hline
		2.   & 2013-11-22 & 0.02 & Damian Bogel & Korekta \\ \hline
	\end{tabular}
\end{table}
\newpage

\tableofcontents
\setcounter{page}{2}

\newpage

\section{Wstęp}
\noindent Niniejszy dokument określa architekturę kompilatora języka
\textsc{Gordon}.

\section{Opis ogólnej architektury kompilatora}
\noindent Kompilator języka \textsc{Gordon} przetwarza plik wejściowy w
następujących fazach:
\begin{itemize}
    \item analiza leksykalna i składniowa [G],
    \item wnioskowanie o typach [O1],
    \item eliminacja martwego kodu [R],
    \item wykrywanie zakleszczeń [D], 
    \item optymalizacja [O2],
    \item generowanie kodu wynikowego [N].
\end{itemize}
W nawiasach kwadratowych podano nazwy poszczególnych modułów.

\subsection{Analiza leksykalna [G]}
\noindent Kod analizatora leksykalnego jest generowany za pomocą narzędzia
\textsc{Flex}. Nietypową częścią leksera jest uwzględnianie znaków
niewidocznych w kodzie źródłowym jako symboli znaczących, ponieważ struktura
blokowa języka \textsc{Gordon} jest zdefiniowana przez poziomy wcięć.  

\subsection{Analiza składniowa [G]}
\noindent Kod analizatora składniowego jest generowany za pomocą narzędzia
\textsc{Bison}.  Wynikiem działania modułu jest abstrakcyjne drzewo wyrażenia
reprezentujące program wejściowy.  Następne fazy modyfikują to drzewo lub
uzupełniają je o dodatkowe informacje.

\subsection{Wnioskowanie o typach [O1]}
\noindent Ponieważ język \textsc{Gordon} nie wymaga od programisty podawania
typów wszystkich zmiennych w kodzie źródłowym, potrzebny jest moduł
wnioskowania o typach, który na podstawie programu odtwarza najogólniejszy typ
danej zmiennej. Następnie sprawdzana jest poprawność adnotacji dotyczących
typów względem rzeczywistego użycia zmiennych w programie.

\subsection{Eliminacja martwego kodu [R]}
\noindent Moduł eliminający martwy kod służy do zmniejszenia rozmiaru pliku
wynikowego przez usunięcie kodu, który nie jest wywoływany w toku wykonania
programu.  Po wykryciu martwego kodu zostanie wypisane stosowne ostrzeżenie dla
programisty.

\subsection{Wykrywanie zakleszczeń [D]} \noindent Moduł służący do wykrywania
zakleszczeń jest najbardziej skomplikowaną częścią projektu kompilatora języka
\textsc{Gordon}. Gdy kompilator nie jest w stanie stwierdzić
obecności lub braku zakleszczeń w kompilowanym programie, zostanie zgłoszony
błąd (opcja domyślna) lub ostrzeżenie.

Dodatkowo, gdy moduł wykryje możliwość zakleszczenia, w niektórych przypadkach
zostanie przedstawiony ślad wykonania, który może prowadzić do zakleszczenia. 

\subsection{Optymalizacja [O2]}
\noindent Optymalizacje są wykonywane przez kompilator na podstawie z
dodatkowych informacji niesionych przez typy języka \textsc{Gordon}, które są
bardziej ekspresywne niż typy języka wynikowego. Optymalizacje niskopoziomowe
(czyli korzystające ze szczegółów architektury docelowej) nie są częścią modułu
optymalizatora -- zadanie to zostanie oddelegowane do kompilatora języka
\textsc{C}.

\subsection{Generowanie kodu wynikowego [N]}
\noindent Generowanie kodu wynikowego jest ostatnią fazą kompilacji. Podczas tej
fazy są usuwane
nadmiarowe informacje o typach, a następnie wykonywane jest tłumaczenie do kodu języka
\textsc{C}, gotowego do dalszej kompilacji, np. za pomocą pakietu \textsc{GCC}.

\section{Kwestie implementacyjne}
\noindent Kompilator języka \textsc{Gordon} jest napisany w języku \textsc{OCaml}. W celu
zmniejszenia czasu implementacji w projekcie zostały użyte bezpłatne biblioteki i programy o otwartym
kodzie źródłowym, takie jak: \textsc{Bison}, \textsc{Flex} i \textsc{make}.

\section{Motywacja decyzji architektonicznych}
\subsection{Wybór języka programowania}
\noindent Dojrzałość języka \textsc{OCaml}, w połączeniu z jego silnym systemem typów,
sprawia, że doskonale nadaje się do implementacji kompilatora. Dzięki tej
decyzji możliwe jest zmniejszenie liczby błędów w kodzie korzystając ze
statycznego wymuszania niezmienników na poziomie typów.

\subsection{Wybór języka docelowego}
\noindent Język \textsc{C} został wybrany jako język docelowy kompilacji z dwu względów:
\begin{itemize}
    \item dostępności kompilatorów języka \textsc{C} na wielu platformach,
    \item możliwości wykorzystania istniejących, sprawdzonych narzędzi (konsolidatory,
        optymalizatory).
\end{itemize}

\pagebreak
\subsection{Podział kompilacji na fazy}
\noindent Z podziału kompilacji na fazy wynikają następujące korzyści:
\begin{itemize}
    \item modularyzacja kodu, a dzięki temu możliwość łatwej rozbudowy,
    \item łatwość oddzielnego testowania faz.
\end{itemize}

\end{document}
