\documentclass{documentation}

\begin{document}
\begin{titlepage}
\begin{center}
Studencka Pracownia Inżynierii Oprogramowania\\
Instytutu Informatyki Uniwersytetu Wrocìawskiego\\[6cm]

Damian Bogel, Łukasz Dąbek\\[1cm]
\textsc{\LARGE Kompilator Języka Gordon}\\[0.5cm]
\textsc{\large Architektura kompilatora}

\vfill
Wrocław 2013 \\[2.5cm]

\end{center}
\end{titlepage}

\newpage
\begin{table}
	\centering
	\caption{Historia zmian w dokumencie}
		\begin{tabular}{|r|c|c|l|l|}
		\hline
		Lp.  & Data       & Nr wersji & Autor                 & Zmiana \\ \hline
		1.   & 2013-10-22 & 0.01 & Łukasz Dąbek & Utworzenie dokumentu \\ \hline
	\end{tabular}
\end{table}
\newpage

\tableofcontents
\setcounter{page}{2}

\newpage

\section{Wstęp}
\noindent Niniejszy dokument nakreśla architekturę kompilatora języka \textsc{Gordon}. Mimo, że
w trakcie implementacji pewne szczegóły mogą ulec zmianie, to główna idea pozostanie
niezmieniona.

\section{Opis ogólnej architektury kompilatora}
\noindent Kompilator języka \textsc{Gordon} będzie przetwarzał plik wejściowy w następujących fazach:
\begin{itemize}
    \item analiza leksykalna,
    \item analiza składniowa,
    \item inferencja i sprawdzanie typów,
    \item eliminacja martwego kodu,
    \item wykrywanie zakleszczeń
    \item optymalizacja,
    \item generacja kodu wynikowego.
\end{itemize}

\subsection{Analiza leksykalna}
\noindent Kod leksera zostanie wygenerowany przy użyciu narzędzia \textsc{Flex}.

\subsection{Analiza składniowa}
\noindent Kod analizatora składniowego zostanie wygenerowany przy użyciu narzędzia \textsc{Bison}.

\subsection{Inferencja i sprawdzanie typów}
\noindent System typów języka Gordon jest oparty na teorii typów Hindleya-Millnera.

\subsection{Eliminacja martwego kodu}
\noindent Zostanią użyte kraty.

\subsection{Wykrywanie zakleszczeń}
\noindent Będzie.

\subsection{Optymalizacja}
\noindent Będą optymalizacje.

\subsection{Generacja kody wynikowego}
\noindent Do C.

\section{Kwestie implementacyjne}
\noindent Kompilator języka \textsc{Gordon} zostanie napisany w języku \textsc{OCaml}.

\section{Motywacja decyzji architektonicznych}
\subsection{Wybór języka programowania}
\noindent Dojrzałość języka \textsc{OCaml}, w połączeniu z jego silnym systemem typów,
czyni go doskonałym wyborem do implementacji kompilatora. Dzięki tej
decyzji możliwe jest zmniejszenie liczby błędów w kodzie dzięki statycznemu
wymuszaniu niezmienników na poziomie typów.

\subsection{Wybór języka docelowego}
\noindent Język \textsc{C} został wybrany jako język docelowy kompilacji z kilku względów:
\begin{itemize}
    \item dostępność kompilatorów języka \textsc{C} na wielu platformach,
    \item możliwość wykorzystania instniejących, sprawdzonych narzędzi (konsolidatory,
        optymalizatory).
\end{itemize}

\subsection{Podział kompilacji na fazy}
\noindent Dzięki podziałowi kompilacji na fazy zyskujemy:
\begin{itemize}
    \item modularyzację kodu, a dzięki temu możliwość łatwej rozbudowy,
    \item łatwość oddzielnego testowania faz.
\end{itemize}

\end{document}
