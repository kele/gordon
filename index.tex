\documentclass{documentation}
\usepackage{syntax}

\begin{document}
\begin{titlepage}
\begin{center}
Studencka Pracownia Inżynierii Oprogramowania\\
Instytutu Informatyki Uniwersytetu Wrocławskiego\\[6cm]

Damian Bogel, Łukasz Dąbek\\[1cm]
\textsc{\LARGE Kompilator Języka Gordon}\\[0.5cm]
\textsc{\large Wykaz dokumentów}

\vfill
Wrocław 2014 \\[2.5cm]

\end{center}
\end{titlepage}

\newpage
\begin{table}
	\centering
    \captionsetup{name=Tabela}
	\caption{Historia zmian w dokumencie}
		\begin{tabular}{|r|c|c|l|l|}
		\hline
		Lp.  & Data       & Nr wersji & Autor                 & Zmiana \\ \hline
		1.   & 2014-01-05 & 0.01 & Łukasz Dąbek & Utworzenie dokumentu \\ \hline
	\end{tabular}
\end{table}
\newpage

\tableofcontents
\setcounter{page}{2}

\newpage

\section{Wykaz dokumentów}

\begin{enumerate}
    \item Damian Bogel, Łukasz Dąbek, \emph{Kompilator języka Gordon. Analiza przypadków użycia.} Wrocław, SPIO IIUWr 2014.
    \item Damian Bogel, Łukasz Dąbek, \emph{Kompilator języka Gordon. Architektura kompilatora.} Wrocław, SPIO IIUWr 2014.
    \item Damian Bogel, Łukasz Dąbek, \emph{Kompilator języka Gordon. Gramatyka języka.} Wrocław, SPIO IIUWr 2014.
    \item Damian Bogel, Łukasz Dąbek, \emph{Kompilator języka Gordon. Harmonogram.} Wrocław, SPIO IIUWr 2014.
    \item Damian Bogel, Łukasz Dąbek, \emph{Kompilator języka Gordon. Oszacowanie kosztów projaktu.} Wrocław, SPIO IIUWr 2014.
    \item Damian Bogel, Łukasz Dąbek, \emph{Kompilator języka Gordon. Podstawowe pojęcia i definicje.} Wrocław, SPIO IIUWr 2014.
    \item Damian Bogel, Łukasz Dąbek, \emph{Kompilator języka Gordon. Przykładowe programy.} Wrocław, SPIO IIUWr 2014.
    \item Damian Bogel, Łukasz Dąbek, \emph{Kompilator języka Gordon. Testy.} Wrocław, SPIO IIUWr 2014.
    \item Damian Bogel, Łukasz Dąbek, \emph{Kompilator języka Gordon. Założenia ogólne.} Wrocław, SPIO IIUWr 2014.
\end{enumerate}

\end{document}
