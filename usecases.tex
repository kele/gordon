\documentclass{documentation}

\begin{document}
\begin{titlepage}
\begin{center}
Studencka Pracownia Inżynierii Oprogramowania\\
Instytutu Informatyki Uniwersytetu Wrocławskiego\\[6cm]

Damian Bogel, Łukasz Dąbek\\[1cm]
\textsc{\LARGE Kompilator Języka Gordon}\\[0.5cm]
\textsc{\large Analiza przypadków użycia}

\vfill
Wrocław 2013 \\[2.5cm]

\end{center}
\end{titlepage}

\newpage
\begin{table}
	\centering
    \captionsetup{name=Tabela,labelsep=period}
	\caption{Historia zmian w dokumencie}
		\begin{tabular}{|r|c|c|l|l|}
		\hline
		Lp.  & Data       & Nr wersji & Autor                 & Zmiana \\ \hline
		1.   & 2013-11-19 & 0.01 & Łukasz Dąbek & Utworzenie dokumentu \\ \hline
	\end{tabular}
\end{table}
\newpage

\tableofcontents
\setcounter{page}{2}

\newpage

\section{Wstęp}
\noindent Celem niniejszego dokumentu jest analiza przypadków użycia kompilatora
języka \textsc{Gordon}. Analiza ta pozwoli na łatwiejsze weryfikowanie wymagań
stawianych kompilatorowi w fazach specyfikacji, implementacji oraz testowania.

\section{Wykaz przypadków użycia}
\noindent Kompilator \textsc{Gordon} może być używany w następujących przypadkach użycia:
\begin{itemize}
    \item kompilacja programu z pliku źródłowego do kodu maszynowego,
    \item podgląd wygenerowanego kodu pośredniego w języku \textsc{C},
    \item sprawdzenie poprawności typów w programie,
    \item sprawdzenie poprawności składniowej programu,
    \item odczyt informacji modułu optymalizatora (dotyczących wykonanych optymalizacji),
    \item sprawdzenie możliwości występowania zakleszczeń w programie (wraz z informacją
        o śladzie wykonania prowadzącym do potencjalnego zakleszczenia),
    \item odczyt informacji o usuniętym martwym kodzie.
\end{itemize}

\section{Komentarz}
\noindent Kompilator języka \textsc{Gordon}, jako produkt przeznaczony dla programistów, jest
obsługiwany inaczej niż typowe aplikacje z graficznym interfejsem użytkownika.
Kompilator nie jest programem interaktywnym -- po uruchomieniu kompilatora z
odpowiednimi argumentami następuje wypisanie odpowiedzi na ekranie (m.in.
informacji o sukcesie lub niepowodzeniu) i zakończenie programu.

Z tego powodu każdy przypadek użycia jest jednokrokowy -- wymusza to sama natura obsługi
kompilatora.

\end{document}
