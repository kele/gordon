\documentclass{documentation}

\begin{document}
\begin{titlepage}
\begin{center}
Studencka Pracownia Inżynierii Oprogramowania\\
Instytutu Informatyki Uniwersytetu Wrocławskiego\\[6cm]

Damian Bogel, Łukasz Dąbek\\[1cm]
\textsc{\LARGE Kompilator Języka Gordon}\\[0.5cm]
\textsc{\large Analiza przypadków użycia}

\vfill
Wrocław 2013 \\[2.5cm]

\end{center}
\end{titlepage}

\newpage
\begin{table}
	\centering
	\caption{Historia zmian w dokumencie}
		\begin{tabular}{|r|c|c|l|l|}
		\hline
		Lp.  & Data       & Nr wersji & Autor                 & Zmiana \\ \hline
		1.   & 2013-11-19 & 0.01 & Łukasz Dąbek & Utworzenie dokumentu \\ \hline
	\end{tabular}
\end{table}
\newpage

\tableofcontents
\setcounter{page}{2}

\newpage

\section{Wstęp}
\noindent Celem niniejszego dokumentu jest analiza przypadków użycia kompilatora
języka \textsc{Gordon}. Analiza ta pozwoli na łatwiejsze weryfikowanie wymagań
stawianych produktowi w fazach specyfikacji, implementacji oraz testowania.

\section{Analiza przypadków użycia}
\subsection{Kompilacja programu}
\begin{itemize}
    \item Uruchomienie kompilatora z argumentem będącym ścieżką do pliku źródłowego.
\end{itemize}

\end{document}
