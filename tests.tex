\documentclass{documentation}

\begin{document}
\begin{titlepage}
\begin{center}
Studencka Pracownia Inżynierii Oprogramowania\\
Instytutu Informatyki Uniwersytetu Wrocławskiego\\[6cm]

Damian Bogel, Łukasz Dąbek\\[1cm]
\textsc{\LARGE Kompilator Języka Gordon}\\[0.5cm]
\textsc{\large Testy}

\vfill
Wrocław 2013 \\[2.5cm]

\end{center}
\end{titlepage}

\newpage
\begin{table}
	\centering
    \captionsetup{name=Tabela,labelsep=period}
	\caption{Historia zmian w dokumencie}
		\begin{tabular}{|r|c|c|l|l|}
		\hline
		Lp.  & Data       & Nr wersji & Autor                 & Zmiana \\ \hline
		1.   & 2013-11-20 & 0.01 & Damian Bogel & Utworzenie dokumentu \\ \hline
		2.   & 2013-11-25 & 0.02 & Łukasz Dąbek & Korekta \\ \hline
	\end{tabular}
\end{table}
\newpage

\tableofcontents
\setcounter{page}{2}

\newpage

\section{Wstęp}
\noindent Dokument zawiera plan testowania kompilatora języka
\textsc{Gordon}.

\section{Uwagi ogólne}
\noindent Każdy programista zostaje zobowiązany do napisania testów
jednostkowych pokrywających jego część pracy. Każdy z testów znajduje sie w
oddzielnym podkatalogu katalogu \emph{tests}. Testy jednostkowe są w pełni
zautomatyzowane. Po ukończeniu każdej z faz, program zostaje sprawdzony przy
ich użyciu.

\section{Główny program testujący}
\noindent Równolegle z implementacją kompilatora, jest tworzony główny program
testowy, w~ którym użyte zostają wszystkie konstrukcje języka \textsc{Gordon}.
Jego objętość i złożoność jest wystarczająco duża, aby mógł być potraktowany
jako wyznacznik poprawności implementacji kompilatora. Powstaje on w iteracjach,
co umożliwia testowanie jego kolejnych wersji. Niedopuszczalne są
jakiekolwiek (nawet najmniejsze) niezgodności tego programu ze specyfikacją.

\section{Testy analizatora leksykalnego i analizatora składniowego [G]}
\noindent Przygotowanych zostaje kilkanaście programów, w których użyte
zostają wszystkie konstrukcje języka \textsc{Gordon}. W tej fazie dopuszczalne
są także kody źródłowe, w~ których wnioskowanie o typach nie może się powieść.
Moduł [G] musi je jednak przyjąć jako poprawne wejście.

\section{Testy modułu wnioskującego o typach [O1]}
\noindent Projekt kompilatora \textsc{Gordon} zostaje zapisany jako
odpowiadający mu zbiór typów. W trakcie implementacji powinien się on zmieniać
razem z architekturą kompilatora, za każdym razem skutkując poprawnym
wnioskowaniem o typach.

\section{Testy modułu eliminującego martwy kod [R]}
\noindent Przygotowanych zostaje kilkanaście małych programów, w których celowo
zostaje umieszczony kod, który nigdy się nie wykona. Moduł [R] powinien
wyeliminować taki kod.

\section{Testy modułu wykrywającego zakleszczenia [D]}
\noindent Uproszczone wersje istniejącego oprogramowania korzystajacego z
wielowątkowości zostają utworzone w języku \textsc{Gordon} i przekazane do
modułu [D]. Każde wykryte zakleszczenie zostaje wnikliwie sprawdzone przez
programistów, celem ustalenia poprawności działania modułu.

Oprócz tego zostaje utworzonych kilka programów, o których wiadomo, że może w
nich wystąpić zakleszczenie. Moduł [D] powinien wykryć każde z nich.

\section{Testy modułu optymalizującego [O2]}
\noindent Wszystkie testy jednostkowe oraz główny program testowy są sprawdzane
pod kątem zgodności ich wersji przed i po optymalizacji. Przez zgodność rozumie
się to samo wejście, wyjście oraz liczbę wykrytych zakleszczeń.

\section{Generowanie kodu wynikowego [N]}
\noindent Kod wynikowy w języku \textsc{C} kompilowany jest przy użyciu
\textsc{GCC} w wersji 4.7.3 z użyciem flag \texttt{-Wall}, \texttt{-Wextra} oraz
\texttt{-O0} lub \texttt{-O2} w celu sprawdzenia poprawności.

\end{document}
