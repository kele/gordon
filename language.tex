\documentclass{documentation}
\usepackage{syntax}

\begin{document}
\begin{titlepage}
\begin{center}
Studencka Pracownia Inżynierii Oprogramowania\\
Instytutu Informatyki Uniwersytetu Wrocławskiego\\[6cm]

Damian Bogel, Łukasz Dąbek\\[1cm]
\textsc{\LARGE Kompilator Języka Gordon}\\[0.5cm]
\textsc{\large Opis języka}

\vfill
Wrocław 2013 \\[2.5cm]

\end{center}
\end{titlepage}

\newpage
\begin{table}
	\centering
    \captionsetup{name=Tabela}
	\caption{Historia zmian w dokumencie}
		\begin{tabular}{|r|c|c|l|l|}
		\hline
		Lp.  & Data       & Nr wersji & Autor                 & Zmiana \\ \hline
		1.   & 2014-01-05 & 0.01 & Damian Bogel & Utworzenie dokumentu \\ \hline
	\end{tabular}
\end{table}
\newpage

\tableofcontents
\setcounter{page}{2}

\newpage

\section{Ogólna charakterystyka języka}

\section{Gramatyka języka}
\begin{grammar}

<program> ::= <decl-list>

<decl> ::= <decl-keyword> <ident> <ident-list> `=' <body>
\alt <type> <uppercase-ident> `=' 
\alt <const-list>

<decl-keyword> ::= `let'
\alt `letshared'
\alt `letatomic'

<constr> ::= <uppercase-ident> <field-list>

<field> ::= <uppercase-ident> <ident>

<body> ::= <expr-list>

<expr> ::= <ident> `:=' <expr>
\alt `if' <expr> `then' <expr> `else' <expr>
\alt `if' <expr> then <expr>
\alt `while' <expr> `do' <expr>
\alt `lock' <body> `unlock'
\alt `lock' <ident>
\alt `unlock' <ident>
\alt `rlock' <body> `runlock'
\alt `rlock' <ident>
\alt `runlock' <ident>
\alt `wlock' <body> `wunlock'
\alt `wlock' <ident>
\alt `wunlock' <ident>
\alt `for' <ident> `in' <expr> `to' <expr> `do' <body>
\alt `let' <ident> `=' <expr>
\alt `go' <body>
\alt `call' <ident> <arg-list>
\alt `break'
\alt `continue'
\alt <literal>

<prim-type> ::= `Int'
\alt `Bool'
\alt `Char'
\alt `Array' <type> <positive-num>
\alt `Ptr' <type>
\alt `Func' <type-list> `-\>' <type>

<arg-list> ::= <expr>
\alt <expr> <arg-list>

<decl-list> ::= <decl>
\alt <decl>`,' <decl-list>

<expr-list> ::= <expr>
\alt <expr>`,' <expr-list>

<field-list> ::= <field>
\alt <field>`,' <field-list>

<type-list> ::= <type>
\alt <type>`,' <type-list>

<literal> ::= dowolny literał występujący w języku C

<ident> ::= ciąg cyfr oraz małych i wielkich liter alfabetu łacińskiego, zaczynający się literą

<uppercase-ident> ::= <ident> zaczynający się wielką literą

\end{grammar}


\section{Opisy konstrukcji językowych}

\subsection{Przypisanie}
Operator przypisania zmienia wartość zmiennej na wartość wyrażenia występującego
po prawej stronie operatora. Przykład:
\begin{verbatim}
x := 34 * 12 + 7
\end{verbatim}

\subsection{Deklaracja lokalna}
W języku \textsc{Gordon} zmienne lokalne deklaruje się przy użyciu słowa kluczoewgo
\texttt{let}. Zmienna obejmuje swoim zasięgiem bieżący blok i jest niszczona po
jego opuszczeniu. Przykład:

\begin{verbatim}
let x = 100
let y = 2*x + 210

z := x + x * y
\end{verbatim}

\subsection{Wyrażenie warunkowe}
Wyrażenie warunkowe ma semantykę analogiczną do wyrażenia \texttt{if}
w języku \textsc{C}. Podobnie, wyrażenie warunkowe może pominąć blok \texttt{else}.
Przykład:

\begin{verbatim}
if x > 10
  then x := x + 2
  else x := x - 3

if x != 100 && y != 200
  then break
\end{verbatim}

\subsection{Pętla \texttt{while}}
Pętla \texttt{while} (dopóki) ma semantykę analogiczną do pętli \texttt{while} w języku
\textsc{C}. Przykład:

\begin{verbatim}
let y = 1
while x > 0 do
  y := y * x
  x := x - 1
\end{verbatim}

\subsection{Pętla \texttt{for}}
Pętla \texttt{for} w języku \textsc{Gordon} jest lukrem syntaktycznym. Schemat tłumaczenia:

\begin{verbatim}
for x in e1 to e2 do e
=
let x = e1
while x <= e2 do
  e
  x := x + 1
\end{verbatim}

\subsection{Wołanie funkcji}
Aby wywołać funkcję należy po jej nazwie dodać listę wyrażeń do przekazania jako agrumenty dla funkcji.
Wartością całego wyrażenia będzie wartość zwrócona przez funkcję.
Przykład:

\begin{verbatim}
let fac n =
  if n < 2
    then 1
    else n * fac (n-1)

let newton n k =
  fac n / (fac k * fac (n-k))
\end{verbatim}

\subsection{Instrukcje \texttt{break} i \texttt{continue}}
Słowa kluczowe \texttt{break} i \texttt{continue} mogą występować jedynie w pętlach
\texttt{while} i \texttt{for}. Ich semantyka jest analogiczna do semantyki tych samych
instrukcji w języku \textsc{C}: \texttt{break} przerywa działanie pętli a \texttt{continue}
natychmiast przechodzi do kolejnego obrotu pętli. Przykład:

\begin{verbatim}
while true do
  x := x - 1
  if x == 0 then break
\end{verbatim}

\subsection{Uruchamianie wątków}
W nowym wątku można uruchomić dowolne wyrażenie poprzedzając je słowem kluczowym
\texttt{go}. W poniższym przykładzie wartość zmiennej \texttt{x} może być
zarówno 2 jak i 3:

\begin{verbatim}
let x = 0
go (x := 2)
go (x := 3)
\end{verbatim}

\end{document}
