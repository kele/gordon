\documentclass{documentation}

\begin{document}
\begin{titlepage}
\begin{center}
Studencka Pracownia Inżynierii Oprogramowania\\
Instytutu Informatyki Uniwersytetu Wrocławskiego\\[6cm]

Damian Bogel, Łukasz Dąbek\\[1cm]
\textsc{\LARGE Kompilator Języka Gordon}\\[0.5cm]
\textsc{\large Opis języka}

\vfill
Wrocław 2013 \\[2.5cm]

\end{center}
\end{titlepage}

\newpage
\begin{table}
	\centering
    \captionsetup{name=Tabela}
	\caption{Historia zmian w dokumencie}
		\begin{tabular}{|r|c|c|l|l|}
		\hline
		Lp.  & Data       & Nr wersji & Autor                 & Zmiana \\ \hline
		1.   & 2014-01-05 & 0.01 & Damian Bogel & Utworzenie dokumentu \\ \hline
	\end{tabular}
\end{table}
\newpage

\tableofcontents
\setcounter{page}{2}

\newpage

\section{Ogólna charakterystyka języka}

\section{Gramatyka języka}

\section{Opisy konstrukcji językowych}

\subsection{Przypisanie}
Operator przypisania zmienia wartość zmiennej na wartość wyrażenia występującego
po prawej stronie operatora. Przykład:
\begin{verbatim}
x := 34 * 12 + 7
\end{verbatim}

\subsection{Deklaracja lokalna}
W języku \textsc{Gordon} zmienne lokalne deklaruje się przy użyciu słowa kluczoewgo
\texttt{let}. Zmienna obejmuje swoim zasięgiem bieżący blok i jest niszczona po
jego opuszczeniu. Przykład:

\begin{verbatim}
let x = 100
let y = 2*x + 210

z := x + x * y
\end{verbatim}

\subsection{Wyrażenie warunkowe}
Wyrażenie warunkowe ma semantykę analogiczną do wyrażenia \texttt{if}
w języku \textsc{C}. Podobnie, wyrażenie warunkowe może pominąć blok \texttt{else}.
Przykład:

\begin{verbatim}
if x > 10
  then x := x + 2
  else x := x - 3

if x != 100 && y != 200
  then break
\end{verbatim}

\subsection{Pętla \texttt{while}}
Pętla \texttt{while} (dopóki) ma semantykę analogiczną do pętli \texttt{while} w języku
\textsc{C}. Przykład:

\begin{verbatim}
let y = 1
while x > 0 do
  y := y * x
  x := x - 1
\end{verbatim}

\subsection{Pętla \texttt{for}}
Pętla \texttt{for} w języku \textsc{Gordon} jest lukrem syntaktycznym. Schemat tłumaczenia:

\begin{verbatim}
for x in e1 to e2 do e
=
let x = e1
while x <= e2 do
  e
  x := x + 1
\end{verbatim}

\subsection{Wołanie funkcji}
Aby wywołać funkcję należy po jej nazwie dodać listę wyrażeń do przekazania jako agrumenty dla funkcji.
Wartością całego wyrażenia będzie wartość zwrócona przez funkcję.
Przykład:

\begin{verbatim}
let fac n =
  if n < 2
    then 1
    else n * fac (n-1)

let newton n k =
  fac n / (fac k * fac (n-k))
\end{verbatim}

\subsection{Instrukcje \texttt{break} i \texttt{continue}}
Słowa kluczowe \texttt{break} i \texttt{continue} mogą występować jedynie w pętlach
\texttt{while} i \texttt{for}. Ich semantyka jest analogiczna do semantyki tych samych
instrukcji w języku \textsc{C}: \texttt{break} przerywa działanie pętli a \texttt{continue}
natychmiast przechodzi do kolejnego obrotu pętli. Przykład:

\begin{verbatim}
while true do
  x := x - 1
  if x == 0 then break
\end{verbatim}

\end{document}
