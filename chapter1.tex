\documentclass[12pt]{article}

\usepackage[polish]{babel}
\usepackage[margin=1.3in]{geometry}
\usepackage{polski}
\usepackage{fontspec}
\defaultfontfeatures{Mapping=tex-text}
\setmainfont{Verdana}
\setsansfont{Verdana}
\setmonofont{Droid Sans Mono}
\frenchspacing

\begin{document}

\begin{titlepage}
\begin{center}
Studencka Pracownia Inżynierii Oprogramowania\\[3.5cm]

Damian Bogel, Łukasz Dąbek\\[1cm]
\textsc{\LARGE Kompilator Języka Gordon}\\[1cm]
\textsc{\large Podstawowe pojęcia i definicje}

\vfill
Wrocław 2013

\end{center}
\end{titlepage}

\newpage
\setcounter{page}{2}
~
\newpage

\section{Definicje}
\begin{description}
	\item[debuger (ang. debugger)] program służący do analizy działania programu
	i wykrywania błędów.
	\item[konsolidator (ang. linker)] 
	\item[procedury wielobieżne (ang. reentrant procedures)] - procedury,
	których wykonywanie może być w pewnym momencie przerwane, a następnie
	bezpiecznie wywołane ponownie.
	\item[uNDEFINED BEHAVIOUR! (nieokreślone zachowanie?)]
	\item[unikanie zakleszczen] - czynnośc polegająca na zmianie działania
	programu tak, aby nie dopuścić do zakleszczenia. (zob. też zakleszczenie,
	wykrywanie zakleszczeń)
	\item[wielozadaniowość] - cecha systemu operacyjnego pozwalająca na
	wykonywanie więcej niż jednego procesu.
	\item[wielowątkowość (ang. multithreading)] - cecha
	\item[współbieżność]
	\item[wykrywanie zakleszczen] - czynność polegająca na wykrywaniu sytuacji,
	w których doszło do zakleszczenia. (zob. też zakleszczenie, unikanie
	zakleszczeń)
	\item[kompilator]
    \item[kompilator] Takie coś, co wypluwa binarkę.
\end{description}

\end{document}
