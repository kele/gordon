\documentclass[12pt]{article}

\usepackage[polish]{babel}
\usepackage[margin=1.3in]{geometry}
\usepackage{polski}
\usepackage{fontspec}
\defaultfontfeatures{Mapping=tex-text}
\setmainfont{Verdana}
\setsansfont{Verdana}
\setmonofont{Droid Sans Mono}
\frenchspacing

\begin{document}

\begin{titlepage}
\begin{center}
Studencka Pracownia Inżynierii Oprogramowania\\[3.5cm]

Damian Bogel, Łukasz Dąbek\\[1cm]
\textsc{\LARGE Kompilator Języka Gordon}\\[1cm]
\textsc{\large Podstawowe pojęcia i definicje}

\vfill
Wrocław 2013

\end{center}
\end{titlepage}

\newpage
\setcounter{page}{2}
~
\newpage

\section{Definicje}
\begin{description}
    \item[Kompilacja] (ang. \emph{compilation}) Proces automatycznego tłumaczenia kodu źródłowego na kod maszynowy.
    \item[Algorytm wzajemnego wykluczania] (ang. \emph{mutex}) Metoda wymuszenia jednoczesnego dostępu do pamięci współdzielonej przez conajwyżej jeden wątek. Zobacz też: \emph{Zakleszczenie}.
    \item[Zakleszczenie] (ang. \emph{deadlock}) Stan w którym dwa wątki lub procesy wzajemnie na siebie oczekują.
    \item[Zagłodzenie] (ang. \emph{starvation}) Stan w którym proces lub wątek nie może dokończyć działania z powodu braku dostępu do potrzebnych zasobów (najczęściej czasu procesora).
    \item[Semafor] (ang. \emph{semaphore}) Abstrakcyjny typ danych chroniący dostępu do wspólnego zasobu w środowisku współbieżnym. Obsługuje dwie operacje: \texttt{wait} (czekaj na zasób) i \texttt{release} (zwolnij zasób).
    \item[Proces] (ang. \emph{process}) Definiowana przez system operacyjny odosobniona jednostka wykonywania kodu. Zobacz też: \emph{wątek}.
    \item[Wątek] (ang. \emph{thread}) Część programu wykonywana współbieżnie w ramach jednego procesu. W obrębie procesu może być wykonywanych wiele wątków.
    \item[Analiza składniowa] (ang. \emph{parsing}) Proces analizy tekstu przeprowadzany w celu ustalenia jego struktury gramatycznej i zgodności z gramatyką języka.
    \item[Kontekst wykonania] (ang. \emph{execution context}) Stan właściwy uruchomionemu procesowi. Składają się na niego m.in.bieżący stan rejestrów, stos, uchwyty operacji wejścia-wyjścia.
\end{description}

\end{document}
