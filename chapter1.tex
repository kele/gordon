\documentclass[12pt]{article}

\usepackage[polish]{babel}
\usepackage[margin=1.3in]{geometry}
\usepackage{polski}
\usepackage{fontspec}
\defaultfontfeatures{Mapping=tex-text}
\setmainfont{Verdana}
\setsansfont{Verdana}
\setmonofont{Droid Sans Mono}
\frenchspacing

\begin{document}

\begin{titlepage}
\begin{center}
Studencka Pracownia Inżynierii Oprogramowania\\[3.5cm]

Damian Bogel, Łukasz Dąbek\\[1cm]
\textsc{\LARGE Kompilator Języka Gordon}\\[1cm]
\textsc{\large Podstawowe pojęcia i definicje}

\vfill
Wrocław 2013

\end{center}
\end{titlepage}

\newpage
\setcounter{page}{2}
~
\newpage

\section{Definicje}
\begin{description}
	\item[debuger] (ang. \emph{debugger}) program służący do analizy działania programu i wykrywania błędów.
	\item[konsolidator] (ang. linker)
	\item[procedury wielobieżne] (ang. \emph{reentrant procedures}) procedury, których wykonywanie może być w pewnym momencie przerwane, a następnie bezpiecznie wywołane ponownie.
	\item[niezdefiniowane zachowanie] (ang. \emph{undefined behaviour})
	\item[unikanie zakleszczen] czynnośc polegająca na zmianie działania programu tak, aby nie dopuścić do zakleszczenia. (zob. też \emph{zakleszczenie}, \emph{wykrywanie zakleszczeń})
	\item[wielowątkowość] (ang. \emph{multithreading}) - cecha system operacyjnego pozwalająca na wykonywanie w ramach jednego procesu wielu wątków, często jednocześnie.
	\item[program wielowątkowy] program korzystający z wielowątkowości systemu operacyjnego.
	\item[wielozadaniowość] cecha systemu operacyjnego pozwalająca na wykonywanie więcej niż jednego zadania.
	\item[przetwarzanie współbieżne] (ang. \emph{concurrent computing})
	\item[przetwarzanie równoległe] (ang. \emph{parallel computing})
	\item[synchronizacja] 
	\item[wykrywanie zakleszczen] czynność polegająca na wykrywaniu sytuacji, w których doszło do zakleszczenia. (zob. też \emph{zakleszczenie}, \emph{unikanie zakleszczeń})
    \item[algorytm wzajemnego wykluczania] Metoda wymuszenia jednoczesnego dostępu do pamięci wpsółdzielonej przez conajwyżej jeden wątek. Zobacz też: \emph{zakleszczenie}.
    \item[kompilacja] (ang. \emph{compilation}) Proces automatycznego tłumaczenia kodu
    \item[kompilator] Takie coś, co wypluwa binarkę.
    \item[zakleszczenie] Stan w którym dwa wątki lub procesy wzajemnie na siebie oczekują.
    źródłowego na kod maszynowy.
\end{description}

\end{document}
