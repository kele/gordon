\documentclass[11pt]{article}

\usepackage[polish]{babel}
\usepackage[margin=1.3in]{geometry}
\usepackage{polski}
\usepackage{fontspec}
\defaultfontfeatures{Mapping=tex-text}
\setmainfont{Verdana}
\setsansfont{Verdana}
\setmonofont{Droid Sans Mono}
\frenchspacing

\begin{document}

\begin{titlepage}
\begin{center}
Studencka Pracownia Inżynierii Oprogramowania\\[6cm]

Damian Bogel, Łukasz Dąbek\\[1cm]
\textsc{\LARGE Kompilator Języka Gordon}\\
\textsc{\large Podstawowe pojęcia i definicje}

\vfill
Wrocław 2013

\end{center}
\end{titlepage}

\newpage
% TODO: Changelog
\setcounter{page}{2}
~
\newpage

\section{Definicje}
Podstawowe pojęcia używane w dokumentacji kompilatora języka Gordon:
\begin{description}
    \item[algorytm wzajemnego wykluczania] (ang. \emph{mutex}) -- metoda wymuszenia jednoczesnego dostępu do pamięci współdzielonej przez conajwyżej jeden wątek; zobacz też: \emph{zakleszczenie};
    \item[analiza składniowa] (ang. \emph{parsing}) -- proces analizy tekstu przeprowadzany w celu ustalenia jego struktury gramatycznej i zgodności z gramatyką języka;
    \item[debuger] (ang. \emph{debugger}) -- program służący do analizy działania programu i wykrywania błędów;
    \item[kompilacja] (ang. \emph{compilation}) -- proces automatycznego tłumaczenia kodu źródłowego na kod maszynowy;
    \item[konsolidator] (ang. linker)
    \item[kontekst wykonania] (ang. \emph{execution context}) -- stan właściwy uruchomionemu procesowi; składają się na niego m.in. bieżący stan rejestrów, stos, uchwyty operacji wejścia-wyjścia;
    \item[niezdefiniowane zachowanie] (ang. \emph{undefined behaviour})
    \item[procedury wielobieżne] (ang. \emph{reentrant procedures}) -- procedury, których wykonywanie może być w pewnym momencie przerwane, a następnie bezpiecznie wywołane ponownie;
    \item[proces] (ang. \emph{process}) -- definiowana przez system operacyjny odosobniona jednostka wykonywania kodu; zobacz też: \emph{wątek};
    \item[program wielowątkowy] -- program korzystający z wielowątkowości systemu operacyjnego;
    \item[przetwarzanie równoległe] (ang. \emph{parallel computing})
    \item[przetwarzanie współbieżne] (ang. \emph{concurrent computing})
    \item[semafor] (ang. \emph{semaphore}) -- abstrakcyjny typ danych chroniący dostępu do wspólnego zasobu w środowisku współbieżnym; obsługuje dwie operacje: \texttt{wait} (czekaj na zasób) i \texttt{release} (zwolnij zasób);
    \item[synchronizacja] czynność polegająca na koordynowaniu dostępu do danych, które są dzielone między wątkami;
    \item[transakcja] TODO
    \item[unikanie zakleszczen] -- czynność polegająca na zmianie działania programu tak, aby nie dopuścić do zakleszczenia; zobacz też \emph{zakleszczenie}, \emph{wykrywanie zakleszczeń};
    \item[wielowątkowość] (ang. \emph{multithreading}) -- cecha system operacyjnego pozwalająca na wykonywanie w ramach jednego procesu wielu wątków, często jednocześnie;
    \item[wielozadaniowość] -- cecha systemu operacyjnego pozwalająca na wykonywanie więcej niż jednego zadania;
    \item[wykrywanie zakleszczen] -- czynność polegająca na wykrywaniu sytuacji, w których doszło do zakleszczenia; zobacz też: \emph{zakleszczenie}, \emph{unikanie zakleszczeń};
    \item[wątek] (ang. \emph{thread}) -- część programu wykonywana współbieżnie w ramach jednego procesu; w obrębie procesu może być wykonywanych wiele wątków;
    \item[zagłodzenie] (ang. \emph{starvation}) -- stan w którym proces lub wątek nie może dokończyć działania z powodu braku dostępu do potrzebnych zasobów (najczęściej czasu procesora);
    \item[zakleszczenie] (ang. \emph{deadlock}) -- stan w którym dwa wątki lub procesy wzajemnie na siebie oczekują.
\end{description}

\end{document}
