\documentclass[12pt]{article}

\usepackage[polish]{babel}
\usepackage[margin=1.3in]{geometry}
\usepackage{polski}
\usepackage{fontspec}
\defaultfontfeatures{Mapping=tex-text}
\setmainfont{Verdana}
\setsansfont{Verdana}
\setmonofont{Droid Sans Mono}
\frenchspacing

\begin{document}

\begin{titlepage}
\begin{center}
Studencka Pracownia Inżynierii Oprogramowania\\[3.5cm]

Damian Bogel, Łukasz Dąbek\\[1cm]
\textsc{\LARGE Kompilator Języka Gordon}\\[1cm]
\textsc{\large Podstawowe pojęcia i definicje}

\vfill
Wrocław 2013

\end{center}
\end{titlepage}

\newpage
\setcounter{page}{2}
~
\newpage

\section{Definicje}
\begin{description}
    \item[Kompilacja] (ang. \emph{compilation}) Proces automatycznego tłumaczenia kodu
    źródłowego na kod maszynowy.
    \item[Algorytm wzajemnego wykluczania] Metoda wymuszenia jednoczesnego dostępu do pamięci
        wpsółdzielonej przez conajwyżej jeden wątek. Zobacz też: \emph{Zakleszczenie}.
    \item[Zakleszczenie] Stan w którym dwa wątki lub procesy wzajemnie na siebie oczekują.
\end{description}

\end{document}
