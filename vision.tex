\documentclass{documentation}

\begin{document}
\begin{titlepage}
\begin{center}
Studencka Pracownia Inżynierii Oprogramowania\\
Instytutu Informatyki Uniwersytetu Wrocławskiego\\[6cm]

Damian Bogel, Łukasz Dąbek\\[1cm]
\textsc{\LARGE Kompilator Języka Gordon}\\[0.5cm]
\textsc{\large Założenia ogólne}

\vfill
Wrocław 2014 \\[2.5cm]

\end{center}
\end{titlepage}

\newpage
\begin{table}
	\centering
    \captionsetup{name=Tabela,labelsep=period}
	\caption{Historia zmian w dokumencie}
		\begin{tabular}{|r|c|c|l|l|}
		\hline
		Lp.  & Data       & Nr wersji & Autor                 & Zmiana \\ \hline
		1.   & 2013-11-05 & 0.01 & Damian Bogel & Utworzenie dokumentu \\ \hline
		2.   & 2013-11-25 & 0.02 & Damian Bogel & Korekta \\ \hline
	\end{tabular}
\end{table}
\newpage

\tableofcontents
\setcounter{page}{2}

\newpage

\section{Wprowadzenie}
\subsection{Wstęp}
\noindent Niniejszy dokument ma na celu określenie przeznaczenia programu
\textsc{Gordon} oraz wyspecyfikowanie stawianych przed nim wymagań.

\subsection{Ogólny opis programu \textsc{Gordon}}
\noindent Program \textsc{Gordon} jest kompilatorem imperatywnego języka
programowania o składni podobnej do języka Haskell, którego główną własnością
są konstrukcje umożliwiające wykrywanie zakleszczeń. Język ten także nazywa się
\textsc{Gordon}.

Celem \textsc{Gordona} jest ułatwienie pracy programistom tworzącym
oprogramowanie współbieżne. Ponieważ praktyka pokazuje, że napisanie poprawnego
programu współbieżnego (tj. wolnego od zakleszczeń, głodzenia) jest zadaniem
trudnym, projekt \textsc{Gordon} uwzględnia szereg usprawnień wspomagających
programistę. Usprawnienia te znajdują się zarówno w samej konstrukcji języka,
jak i w statycznej analizie kodu będącej częścią kompilatora.

\section{Grupa docelowa}
\subsection{Opis użytkowników}
\noindent Użytkownikami kompilatora \textsc{Gordon} są programiści, którzy
tworzą oprogramowanie współbieżne. Są oni zaznajomieni z takimi pojęciami, jak
wzajemne wykluczanie lub głodzenie.

Od użytkowników wymaga się znajomości przynajmniej jednego imperatywnego języka
programowania. Ze względu na prostotę języka \textsc{Gordon}, można się go
nauczyć w bardzo krótkim czasie. Pozwoli to na korzystanie z kompilatora bez
potrzeby wcześniejszych drogich i czasochłonnych szkoleń.

\textsc{Gordon} jest kompilatorem, który ma wspomagać pracę programisty w
wykrywaniu zakleszczeń i innych problemów związanych z programowaniem
współbieżnym. Od użytkowników nie wymaga się, aby znali formalne metody
wykrywania wymienionych zagrożeń. Wszelkie komunikaty przedstawiane przez
kompilator \textsc{Gordon} będą abstrahować od szczegółów implementacyjnych
tego narzędzia.

\subsection{Potrzeby użytkowników}
\noindent Podstawowymi potrzebami użytkowników kompilatora \textsc{Gordon} są:
\begin{itemize}
\item wspomaganie podstawowych technik programowania współbieżnego (wzajemne
wykluczenia, pamięć dzielona itd.),
\item wykrywanie potencjalnych zakleszczeń,
\item czytelne i zwięzłe komunikaty,
\item prosty język,
\item licencja pozwalająca na szerokie wykorzystywanie.
\end{itemize}

\section{Opis programu}
\subsection{Zastosowania}
\noindent Język \textsc{Gordon} jest narzędziem służącym do tworzenia
oprogramowania wydajnego, współbieżnego oraz bezpiecznego. 
\textsc{Gordon} został opracowany z myślą o:
\begin{itemize}
    \item oprogramowaniu medycznym,
    \item rozproszonym oprogramowaniu baz danych,
    \item systemach czasu rzeczywistego.
\end{itemize}

\subsection{Pozycja na rynku}
\noindent W chwili projektowania kompilatora języka \textsc{Gordon} na rynku
dominowały następujące metody wytwarzania szybkiego i niezawodnego
oprogramowania współbieżnego:
\begin{itemize}
    \item \emph{Metoda wprawnego programisty} -- napisanie programu zleca się
    wprawnemu programiście języka \textsc{C}.
    \item \emph{Metoda formalisty} -- dysponując modelem teoretycznym maszyny,
    odtwarzamy program z jego formalnej specyfikacji.
    \item \emph{Pamięć transakcyjna} -- zastosowanie techniki programowania,
    która nie korzysta z muteksów i umożliwia unikanie zakleszczeń.
\end{itemize}
~\\
Język \textsc{Gordon} ma szansę stać się praktycznym narzędziem tworzenia
bezpiecznego oprogramowania współbieżnego.

\section{Cechy programu}
\noindent Podstawowymi cechami języka \textsc{Gordon} są:
\begin{itemize}
\item imperatywność,
\item składnia wzorowana na języku Haskell,
\item silne typy, statyczne wnioskowanie o typach,
\item podstawowe konstrukcje programowania współbieżnego (wzajemne wykluczenia,
pamięć dzielona itd.),
\item wykrywanie zakleszczeń na podstawie statycznej analizy kodu,
\item generowanie kodu w języku C,
\item przenośność.
\end{itemize}

\section{Dokumentacja}
\subsection{Język \textsc{Gordon}}
\noindent Specyfikacja języka \textsc{Gordon} dostępna jest w oddzielnym
dokumencie\footnote{Damian Bogel, Łukasz Dąbek, \emph{Język \textsc{Gordon}.}
Wrocław, SPIO IIUWr 2014. }. Kompilator \textsc{Gordon} jest pełną
implementacją tego języka.

\subsection{Specyfikacja kompilatora \textsc{Gordon}}
\noindent Specyfikacja kompilatora zawiera:
\begin{itemize}
\item opis wejścia i wyjścia,
\item opis dostępnych opcji kompilatora.
\end{itemize}

\section{Wymagania}
\subsection{Wymagania systemowe}
\noindent System oparty na jądrze \textsc{Linux} udostępniający interfejs
\textsc{pthreads} oraz kompilator języka C.

\subsection{Wymagania sprzętowe}
\noindent Takie same jak w wypadku dowolnego systemu opartego na jądrze
\textsc{Linux}\footnote{Zob. \url{https://www.kernel.org}}.

\subsection{Wymagania wydajnościowe}
\noindent Wykrywanie zakleszczeń odbywa się na poziomie analizy kodu
źródłowego, nie w trakcie wykonania programu. W związku z tym, nie powstaje
żaden narzut wydajnościowy. Różnice między czasami wykonania programu
napisanego w C, a kodem wygenerowanym przez kompilator \textsc{Gordon} powinny
być niezauważalne (poniżej 2\%).

\section{Licencja}
\noindent Kompilator języka \textsc{Gordon} jest objęty licencją GPL w wersji
trzeciej. Jej pełna treść jest dostępna w oddzielnym dokumencie\footnote{Zob.
\url{http://www.gnu.org/licenses/gpl-3.0.txt}}.

\end{document}
