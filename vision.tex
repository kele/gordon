\documentclass{documentation}

\begin{document}
\begin{titlepage}
\begin{center}
Studencka Pracownia Inżynierii Oprogramowania\\
Instytutu Informatyki Uniwersytetu Wrocìawskiego\\[6cm]

Damian Bogel, Łukasz Dąbek\\[1cm]
\textsc{\LARGE Kompilator Języka Gordon}\\[0.5cm]
\textsc{\large Założenia ogólne}

\vfill
Wrocław 2013 \\[2.5cm]

\end{center}
\end{titlepage}

\newpage
\begin{table}
	\centering
	\caption{Historia zmian w dokumencie}
		\begin{tabular}{|r|c|c|l|l|}
		\hline
		Lp.  & Data       & Nr wersji & Autor                 & Zmiana \\ \hline
		1.   & 2013-11-05 & 0.01 & Damian Bogel & Utworzenie dokumentu \\ \hline
	\end{tabular}
\end{table}
\newpage

\tableofcontents
\setcounter{page}{2}

\newpage

\section{Wprowadzenie}
\subsection{Wstęp}
\noindent Niniejszy dokument ma na celu określenie przeznaczenia programu \textsc{Gordon} oraz wyspecyfikowanie stawianych przed nim wymagań.

\subsection{Ogólny opis programu \textsc{Gordon}}
\noindent Program \textsc{Gordon} jest kompilatorem imperatywnego języka  programowania o składni podobnej do języka Haskell, którego główną własnością są konstrukcje pozwalające na wykrywanie zakleszczeń. Język ten także nazywa się \textsc{Gordon}.
Celem \textsc{Gordona} jest ułatwienie pracy programistom tworzącym oprogramowanie wielowątkowe. TODO: jak napisac ze ostatnio coraz wiecej rdzeni, chmury i w ogole duzo?

\section{Grupa docelowa}
\subsection{Opis użytkowników}
DEVELOPERS, DEVELOPERS, DEVELOPERS, DEVELOPERS!
\subsection{Potrzeby użytkowników}
TODO: multithreading and pizza and coffee (optional)

\section{Opis programu}
\subsection{Zastosowania}
TODO: jak ktos napisze duzy projekt bez zakleszczen w pierwszym podejsciu, to stawiam browara. Zastosowaniem jest to, ze moge wygrywac browary. :)
\subsection{Pozycja na rynku}
TODO: mozna wspomniec o jakichs systemach dowodzenia, i ze nikt ich nie zna poza DaBim, Spławskim, Jedynakiem i Sznurkiem

\section{Cechy programu}
TODO: nie wiem co tutaj wpisac

\section{Dokumentacja}
TODO: kompilator nie wymaga duzej dokumentacji, ale na pewno specyf jezyka bedzie trza

\section{Wymagania}
\subsection{Wymagania sprzętowe}
TODO: no ma smigac wszedzie tam gdzie C :))

\subsection{Wymagania wydajnościowe}
TODO: ma byc szybko, bo nie po to cisniemy paralelizm zeby bylo wolno

\section{Licencja}
TODO: GPL?

\end{document}
