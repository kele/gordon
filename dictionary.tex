\documentclass{documentation}
\begin{document}

\begin{titlepage}
\begin{center}
Studencka Pracownia Inżynierii Oprogramowania\\
Instytutu Informatyki Uniwersytetu Wrocławskiego\\[6cm]

Damian Bogel, Łukasz Dąbek\\[1cm]
\textsc{\LARGE Kompilator Języka Gordon}\\[0.5cm]
\textsc{\large Podstawowe pojęcia i definicje}

\vfill
Wrocław 2013 \\[2.5cm]

\end{center}
\end{titlepage}

\newpage
\begin{table}
	\centering
    \captionsetup{name=Tabela}
	\caption{Historia zmian w dokumencie}
		\begin{tabular}{|r|c|c|l|l|}
		\hline
		Lp.  & Data       & Nr wersji & Autor                 & Zmiana \\ \hline
		1.   & 2013-10-22 & 0.01 & Łukasz Dąbek & Utworzenie dokumentu \\ \hline
        2.   & 2013-10-22 & 0.02 & Damian Bogel & Dodanie definicji \\ \hline
	\end{tabular}
\end{table}
\newpage

\tableofcontents
\setcounter{page}{2}

\newpage

\section{Definicje}
\noindent Podstawowe pojęcia używane w dokumentacji kompilatora języka Gordon:
\\
\begin{description}
    \item[algorytm wzajemnego wykluczania] (ang. \emph{mutex}) -- metoda wymuszenia dostępu do pamięci dzielonej przez co najwyżej jeden wątek; zobacz też: \emph{zakleszczenie};\\
    \item[analiza składniowa] (ang. \emph{parsing}) -- proces analizy tekstu wykonywany w celu ustalenia jego struktury gramatycznej i zgodności z gramatyką języka;\\
    \item[debuger] (ang. \emph{debugger}) -- program służący do analizy działania programu i wykrywania błędów;\\
    \item[kompilacja] (ang. \emph{compilation}) -- proces automatycznego tłumaczenia kodu źródłowego na kod maszynowy;\\
    \item[kontekst wykonania] (ang. \emph{execution context}) -- stan właściwy uruchomionemu procesowi; składają się na niego m.in. bieżący stan rejestrów, stos, uchwyty operacji wejścia-wyjścia;\\
    \item[procedury wznawialne] (ang. \emph{reentrant procedures}) -- procedury, których wykonywanie może być w pewnym momencie przerwane, a następnie bezpiecznie wywołane ponownie;\\
    \item[proces] (ang. \emph{process}) -- definiowana przez system operacyjny odosobniona jednostka wykonywania kodu wraz z zasobami do niej należącymi; zobacz też: \emph{wątek};\\
    \item[program wielowątkowy] -- program korzystający z wielowątkowości systemu operacyjnego;\\
    \item[semafor] (ang. \emph{semaphore}) -- abstrakcyjny typ danych służący do ochrony dostępu do wspólnego zasobu w środowisku współbieżnym z dwiema operacjami: \texttt{wait} (czekaj na zasób) i \texttt{release} (zwolnij zasób);\\
    \item[synchronizacja] -- czynność polegająca na koordynowaniu dostępu do danych, które są dzielone między wątkami;\\
\newpage
    \item[transakcja, operacja atomowa] -- działanie, którego wynik powinien pozostać niezależny od pozostałych instrukcji, które działają w tym samym czasie; operacje takie wykonują się jako spójna całość; \\
    \item[unikanie zakleszczen] -- czynność polegająca na zmianie działania programu tak, aby nie dopuścić do zakleszczenia; zobacz też \emph{zakleszczenie}, \emph{wykrywanie zakleszczeń};\\
    \item[wielowątkowość] (ang. \emph{multithreading}) -- cecha system operacyjnego pozwalająca na wykonywanie w ramach jednego procesu wielu wątków, często jednocześnie;\\
    \item[wielozadaniowość] -- cecha systemu operacyjnego pozwalająca na wykonywanie więcej niż jednego zadania;\\
    \item[wykrywanie zakleszczen] -- czynność polegająca na wykrywaniu sytuacji, w których doszło do zakleszczenia; zobacz też: \emph{zakleszczenie}, \emph{unikanie zakleszczeń};\\
    \item[wątek] (ang. \emph{thread}) -- część programu wykonywana współbieżnie w ramach jednego procesu; w obrębie procesu może być wykonywanych wiele wątków;\\
    \item[zagłodzenie] (ang. \emph{starvation}) -- stan w którym proces lub wątek nie może dokończyć działania z powodu braku dostępu do potrzebnych zasobów (najczęściej czasu procesora);\\
    \item[zakleszczenie] (ang. \emph{deadlock}) -- stan w którym dwa wątki lub procesy wzajemnie na siebie oczekują.\\
\end{description}

\end{document}
