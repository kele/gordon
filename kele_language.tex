\documentclass{documentation}
\usepackage{syntax}

\begin{document}
\begin{titlepage}
\begin{center}
Studencka Pracownia Inżynierii Oprogramowania\\
Instytutu Informatyki Uniwersytetu Wrocławskiego\\[6cm]

Damian Bogel, Łukasz Dąbek\\[1cm]
\textsc{\LARGE Kompilator Języka Gordon}\\[0.5cm]
\textsc{\large Opis języka}

\vfill
Wrocław 2014 \\[2.5cm]

\end{center}
\end{titlepage}

\newpage
\begin{table}
	\centering
    \captionsetup{name=Tabela,labelsep=period}
	\caption{Historia zmian w dokumencie}
		\begin{tabular}{|r|c|c|l|l|}
		\hline
		Lp.  & Data       & Nr wersji & Autor                 & Zmiana \\ \hline
		1.   & 2014-01-05 & 0.01 & Damian Bogel & Utworzenie dokumentu \\ \hline
	\end{tabular}
\end{table}
\newpage

\tableofcontents
\setcounter{page}{2}

\newpage

\section{Ogólna charakterystyka języka}

\section{Gramatyka języka}
\begin{grammar}

<program> ::= <decl-list>

<decl> ::= <decl-keyword> <ident> <ident-list> `=' <body>
\alt <type> <uppercase-ident> `=' 
\alt <const-list>

<decl-keyword> ::= `let'
\alt `letshared'
\alt `letatomic'

<constr> ::= <uppercase-ident> <field-list>

<field> ::= <uppercase-ident> <ident>

<body> ::= <expr-list>

<expr> ::= <ident> `:=' <expr>
\alt `if' <expr> `then' <expr> `else' <expr>
\alt `if' <expr> then <expr>
\alt `while' <expr> `do' <expr>
\alt `lock' <body> `unlock'
\alt `lock' <ident>
\alt `unlock' <ident>
\alt `rlock' <body> `runlock'
\alt `rlock' <ident>
\alt `runlock' <ident>
\alt `wlock' <body> `wunlock'
\alt `wlock' <ident>
\alt `wunlock' <ident>
\alt `for' <ident> `in' <expr> `to' <expr> `do' <body>
\alt `let' <ident> `=' <expr>
\alt `go' <body>
\alt `call' <ident> <arg-list>
\alt `break'
\alt `continue'
\alt <literal>

<prim-type> ::= `Int'
\alt `Bool'
\alt `Char'
\alt `Array' <type> <positive-num>
\alt `Ptr' <type>
\alt `Func' <type-list> `-\>' <type>

<arg-list> ::= <expr>
\alt <expr> <arg-list>

<decl-list> ::= <decl>
\alt <decl>`,' <decl-list>

<expr-list> ::= <expr>
\alt <expr>`,' <expr-list>

<field-list> ::= <field>
\alt <field>`,' <field-list>

<type-list> ::= <type>
\alt <type>`,' <type-list>

<literal> ::= dowolny literał występujący w języku C

<ident> ::= ciąg cyfr oraz małych i wielkich liter alfabetu łacińskiego, zaczynający się literą

<uppercase-ident> ::= <ident> zaczynający się wielką literą

\end{grammar}

\section{Opisy konstrukcji językowych}

\end{document}
